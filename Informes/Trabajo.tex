\documentclass[12pt]{report}
\def\TITULO{Título}
\def\subtitulo{Subtítulo}
\def\autora{Autor}
\def\correoa{correo@alumnos.ulagos.cl}
\def\autorb{Autor}
\def\correob{correo@alumnos.ulagos.cl}
\def\autorc{Autor}
\def\correoc{correo@alumnos.ulagos.cl}
\def\autord{Autor}
\def\correod{correo@alumnos.ulagos.cl}
\def\asignatura{Asignatura}
\def\campus{Campus ???}
\def\carrera{Ingeniería Civil en Informática}
%%%%%%%%%%%%%%%%%%%%%%%%%%%%%%%%%%%%%%%%%
% a0poster Portrait Poster
% LaTeX Template
% Version 1.0 (22/06/13)
%
% The a0poster class was created by:
% Gerlinde Kettl and Matthias Weiser (tex@kettl.de)
% 
% This template has been downloaded from:
% http://www.LaTeXTemplates.com
%
% License:
% CC BY-NC-SA 3.0 (http://creativecommons.org/licenses/by-nc-sa/3.0/)
%
%%%%%%%%%%%%%%%%%%%%%%%%%%%%%%%%%%%%%%%%%

%----------------------------------------------------------------------------------------
%	PACKAGES AND OTHER DOCUMENT CONFIGURATIONS
%----------------------------------------------------------------------------------------


%\usepackage[papersize={1000mm,1500mm},tmargin=70mm,bmargin=70mm,lmargin=70mm,rmargin=30mm]{geometry}%comentar esta linea si desean usar solo tamaño A0, esto es para tamaño 150x100
\usepackage[utf8]{inputenc}
\usepackage[T1]{fontenc}
\usepackage[spanish,activeacute,es-tabla]{babel}
\usepackage{multicol} % This is so we can have multiple columns of text side-by-side
\columnsep=100pt % This is the amount of white space between the columns in the poster
\columnseprule=3pt % This is the thickness of the black line between the columns in the poster

\usepackage[svgnames]{xcolor} % Specify colors by their 'svgnames', for a full list of all colors available see here: http://www.latextemplates.com/svgnames-colors
\usepackage{mwe}
\usepackage[absolute]{textpos}
\usepackage[default]{roboto}
%\usepackage{times} % Use the times font
%\usepackage{palatino} % Uncomment to use the Palatino font
\usepackage{graphicx} % Required for including images
\graphicspath{{images/}} % Location of the graphics files
\usepackage{booktabs} % Top and bottom rules for table
\usepackage[font=small,labelfont=bf]{caption} % Required for specifying captions to tables and figures
\usepackage{cite,cancel,fancyvrb,textcomp,times,booktabs,amssymb,amsmath,ragged2e,float,subfig,xspace,epic,eepic,multicol,multirow,colortbl,color,url,hyperref,pict2e,array,listings,pgfpages,eurosym,wasysym,textcase,datetime,amsthm,amsfonts} % For math fonts, symbols and environments
%\usepackage{multicol,graphicx,fancyhdr,eso-pic,url,float,lmodern,listings,times,textcomp, amsthm,amsmath,amssymb,dsfont,color,colortbl,sidecap,xspace,epic,eepic,anysize,setspace, hyperref, multirow,algorithm,algpseudocode,enumitem,pdflscape,lscape,subfigure,csquotes}
\usepackage{wrapfig} % Allows wrapping text around tables and figures
%Colores Ulagos
%COLOREAR SISTEMA
\definecolor{gray97}{gray}{.97}
\definecolor{gray75}{gray}{.75}
\definecolor{gray45}{gray}{.45}
\definecolor{listinggray}{gray}{0.9}
\definecolor{lbcolor}{rgb}{0.9,0.9,0.9}
\definecolor{amarillo}{RGB}{255,183,27}
\definecolor{amarilloc}{RGB}{250,223,141}
\definecolor{verde}{RGB}{118,188,33}
\definecolor{verdec}{RGB}{172,219,144}
\definecolor{rojo}{RGB}{202,54,37}
\definecolor{rojoc}{RGB}{255,176,192}
\definecolor{azul}{RGB}{0,61,166}
\definecolor{celeste}{RGB}{143,199,232}
\definecolor{negro}{RGB}{35,31,32}
\definecolor{naranjo}{RGB}{255,103,29}
\definecolor{naranjoc}{RGB}{255,164,136}
\definecolor{morado}{RGB}{126,87,197}
\definecolor{moradoc}{RGB}{220,168,226}
\definecolor{gris}{RGB}{183,177,169}
\definecolor{grisc}{RGB}{216,209,202}
%COLOREAR TEXTO
\newcommand\rojo[1]{\textcolor[RGB]{202,54,37}{#1}}
\newcommand\rojoc[1]{\textcolor[RGB]{255,176,192}{#1}}
\newcommand\gris[1]{\textcolor[RGB]{183,177,169}{#1}}
\newcommand\grisc[1]{\textcolor[RGB]{216,209,202}{#1}}
\newcommand\azul[1]{\textcolor[RGB]{0,61,166}{#1}}
\newcommand\celeste[1]{\textcolor[RGB]{143,199,232}{#1}}
\newcommand\verde[1]{\textcolor[RGB]{118,188,33}{#1}}
\newcommand\verdec[1]{\textcolor[RGB]{172,219,144}{#1}}
\newcommand\naranjo[1]{\textcolor[RGB]{255,103,29}{#1}}
\newcommand\naranjoc[1]{\textcolor[RGB]{255,164,136}{#1}}
\newcommand\amarillo[1]{\textcolor[RGB]{255,183,27}{#1}}
\newcommand\amarilloc[1]{\textcolor[RGB]{250,223,141}{#1}}
\newcommand\morado[1]{\textcolor[RGB]{126,87,197}{#1}}
\newcommand\moradoc[1]{\textcolor[RGB]{220,168,226}{#1}}
\newcommand\negro[1]{\textcolor[RGB]{35,31,32}{#1}}

\providecommand{\keywords}[1]{\textbf{\textit{Palabras Clave---}} #1}
\newtheorem{ejemplo}{Ejemplo}
\newtheorem{solucion}{Solución}
\newtheorem{definir}{Definición}
\newtheorem{prueba}{Prueba}
\newtheorem{demo}{Demostración}
\newtheorem{obs}{Observación}



%%%CODIGOS DE PROGRAMACION
\lstset{%backgroundcolor=\color{lbcolor},
	frame=Ltb, framerule=0pt, aboveskip=0.5cm, tabsize=4, rulecolor=, %%%CAMBIAR POR LENGUAJE DE PREFERENCIA
	stringstyle=\ttfamily,  %basicstyle=\footnotesize,
	upquote=true, aboveskip={1.5\baselineskip}, columns=fixed, showstringspaces=false, extendedchars=true,breaklines=true, prebreak = \raisebox{0ex}[0ex][0ex]{\ensuremath{\hookleftarrow}}, showtabs=false, showspaces=false, showstringspaces=false,
	%tipos de letra y colores
	identifierstyle=\ttfamily,
	keywordstyle=\bfseries  \color[RGB]{0,2,216}, %palabras reservadas
	commentstyle= \scriptsize\color[rgb]{0,.5,0.2}, %comentarios
	stringstyle=\color{rojo},%cadena de texto
	%numeracion de lineas
	framexleftmargin=0.1cm,%framextopmargin=1pt, framexbottommargin=1pt,
	aboveskip=2.8mm,belowskip=-1mm,
	framesep=0pt, rulesep=.4pt, rulesepcolor=\color{gray75}, numbers=left, numbersep=15pt, numberstyle=\tiny, numberfirstline = false, breaklines=true,literate={á}{{\'a}}1 {é}{{\'e}}1 {í}{{\'i}}1 {ó}{{\'o}}1 {ú}{{\'u}}1
	{Á}{{\'A}}1 {É}{{\'E}}1 {Í}{{\'I}}1 {Ó}{{\'O}}1 {Ú}{{\'U}}1
	{à}{{\`a}}1 {è}{{\`e}}1 {ì}{{\`i}}1 {ò}{{\`o}}1 {ù}{{\`u}}1
	{À}{{\`A}}1 {È}{{\'E}}1 {Ì}{{\`I}}1 {Ò}{{\`O}}1 {Ù}{{\`U}}1
	{ä}{{\"a}}1 {ë}{{\"e}}1 {ï}{{\"i}}1 {ö}{{\"o}}1 {ü}{{\"u}}1
	{Ä}{{\"A}}1 {Ë}{{\"E}}1 {Ï}{{\"I}}1 {Ö}{{\"O}}1 {Ü}{{\"U}}1
	{â}{{\^a}}1 {ê}{{\^e}}1 {î}{{\^i}}1 {ô}{{\^o}}1 {û}{{\^u}}1
	{Â}{{\^A}}1 {Ê}{{\^E}}1 {Î}{{\^I}}1 {Ô}{{\^O}}1 {Û}{{\^U}}1
	{œ}{{\oe}}1 {Œ}{{\OE}}1 {æ}{{\ae}}1 {Æ}{{\AE}}1 {ß}{{\ss}}1
	{ű}{{\H{u}}}1 {Ű}{{\H{U}}}1 {ő}{{\H{o}}}1 {Ő}{{\H{O}}}1
	{ç}{{\c c}}1 {Ç}{{\c C}}1 {ø}{{\o}}1 {å}{{\r a}}1 {Å}{{\r A}}1
	{€}{{\EUR}}1 {£}{{\pounds}}1 {Ñ}{{\~N}}1 {ñ}{{\~n}}1 {¿}{{?`}}1
}
\renewcommand{\lstlistingname}{Código}

\lstset{language=Python}

\begin{document}
	
	%%%%%%%%%%%PORTADA%%%%%%%%%%%%%%%%%%%%%
%\setlength{\unitlength}{1 cm} %Especificar unidad de trabajo
\maketitle

\cleardoublepage
\pagenumbering{roman}
\setcounter{page}{1}

%INDICE GENERAL
\tableofcontents
%INDICE DE FIGURAS
\listoffigures
%INDICE DE TABLAS
\renewcommand{\listtablename}{Índice de tablas}\listoftables
\renewcommand{\lstlistlistingname}{Índice de algoritmos}
\lstlistoflistings
%\addcontentsline
%%%%%%%%%%%%%FIN PORTADA%%%%%%%%%%%%%%%%

%\thispagestyle{empty}
%\begin{abstract}
%\end{abstract}
\cleardoublepage
\pagenumbering{arabic}
\setcounter{page}{1}






%%%%%%%RESUMEN%%%%%%%%%%%
\begin{abstract}\thispagestyle{empty}

Resume en un (1) párrafo el contenido del informe en un máximo de 350 palabras.
Debe ser preciso:
\begin{itemize}\justifying
  \item Establece el problema
  \item Dice porqué es interesante
  \item Señala los logros y desafíos
\end{itemize}
Un resumen debe ser llamativo, motivador, descriptivo y sin contenido específico. \textbf{No incluye}: citas, referencias, conclusiones, figuras ni tablas.



\keywords{Palabra1, Palabra2, Palabra3, Palabra4, Palabra5}
\end{abstract}

\cleardoublepage
\pagenumbering{arabic}
\setcounter{page}{1}






%%%%%%%%COMIENZO


\chapter{Generalidades}
Es lo que comúnmente se conoce como introducción, conduce al lector desde un tema de un área general hacia un campo de investigación específico, describe el contexto, el problema, motiva al lector.

Introduce la terminología, destaca las contribuciones del documento y da una breve descripción de la organización de éste.

Ejemplo de uso de una referencia \cite{001}. Ejemplo de referencia doble \cite{001,002}.

\section{Origen del Tema}
Contextualiza el trabajo respecto de investigaciones previas de otros autores y propias, señala las diferencias con trabajos previos. Algunas veces se incluye en la introducción o bien en la discusión del trabajo (secciones finales). Largo aproximado: 2 páginas.
\section{Planteamiento}

Provee un \naranjo{marco de referencia} para interpretar los resultados y conectarlos a la literatura existente sobre el fenómeno, orienta sobre cómo se realizará el estudio.

 Ayuda a prevenir errores que se han cometido en otros estudios, conduce al establecimiento de la hipótesis o afirmaciones que se someterán a prueba.
 
 Amplia el horizonte del estudio y centra al investigador en el problema, para evitar desviaciones del planteamiento original.

Considera una \naranjo{revisión bibliográfica} que consiste en detectar, obtener y consultar la bibliografía y otros materiales que pueden ser útiles para los propósitos del estudio.

La revisión bibliográfica debe ser selectiva, se puede realizar a partir de tres fuentes principales:

\begin{itemize}\justifying
  \item \naranjo{Primarias (directas):} Libros, artículos, antologías, tesis, disertaciones, entre otros.
  \item \naranjo{Secundarias:} Compilaciones, resúmenes de listados de referencias publicadas en un área en particular, bases de datos.
  \item \naranjo{Terciarias:} Documentos que reúnen nombres y títulos de revistas y otras publicaciones.
\end{itemize}


\begin{ejemplo}
\lipsum[1]%reemplazar esta linea
\end{ejemplo} 


\section{Árbol de Problemas}
\lipsum[1]%reemplazar esta linea

\section{Justificación y Aporte}
Justificar la conveniencia del proyecto desde diversos puntos de vista.

Preguntas clave:
  \begin{itemize}
  \item ¿Para qué sirve la investigación?
  \item ¿Quiénes se benefician con los resultados?
  \item ¿Ayuda a resolver algún problema práctico?
  \item ¿Contribuye a aumentar el conocimiento?
  \item ¿Se podrán generalizar los resultado?
\end{itemize}


\begin{ejemplo}
\lipsum[1]%reemplazar esta linea
\end{ejemplo}


\section{Viabilidad}
Analizar la disponibilidad de recursos financieros, humanos y materiales.

Preguntas clave:
  \begin{itemize}\justifying
  \item ¿Puede llevarse a cabo esta investigación?
  \item ¿Cuánto tiempo tomará realizarla?
\end{itemize}


\section{Alcance}
Que se planea realizar y hasta que punto se espera llegar.

Esta subdivisión debe:
\begin{enumerate}\justifying
  \item Identifique el producto del software para ser diseñado por el nombre (por ejemplo, Anfitrión DBMS, el Generador del Reporte, etc.);
  \item Explique eso que el producto (del software hará y que no hará.
  \item Describe la aplicación del software especificándose los beneficios pertinentes, objetivos, y metas;
  \item Sea consistente con las declaraciones similares en las especificaciones de niveles superiores (por ejemplo, las especificaciones de los requisitos del sistema), si ellos existen.
\end{enumerate}






\chapter{Fundamentación}

\section{Objetivos}\label{objetivos}

Se deben abordar desde el principio de la investigación, expresan los fines que se esperan lograr con el estudio del problema planteado, responden a la pregunta \naranjo{¿Para qué se lleva a cabo la investigación?}, por lo general comienzan con un verbo en infinitivo: Determinar, identificar, establecer, distinguir, medir, cuantificar, entre otros.

Deben enunciar un resultado unívoco, preciso, factible y medible. Su formulación debe ser clara, concisa y bien orientada hacia el fin, en función de ellos se plantean los métodos de recolección de datos, pruebas estadísticas, entre otros.

Evitar unir objetivos, idealmente, un objetivo general y varios específicos.

Cada objetivo específico se ``mapea'' a una pregunta de investigación.
Por ejemplo:
  \begin{itemize}
  \item \textbf{\naranjo{Objetivo:}} Optimizar los métodos de acceso a disco.
  \item \textbf{\naranjo{Preguntas de investigación:}} ¿Cuáles son los métodos de acceso a disco?
\end{itemize}
\subsection{General}
\lipsum[1]%reemplazar esta linea

\subsection{Específicos}
\begin{enumerate}\justifying
  \item \lipsum[1]%reemplazar esta linea
 
  \item \lipsum[1]%reemplazar esta linea

\end{enumerate}



\section{Metodología}
Esto no es hacer referencia a métodos y herramientas que se usarán en el desarrollo del trabajo. Sino que describir como se llevará a cabo el trabajo.

Por lo tanto, nuevamente se puede plantear la solución (el proyecto) en términos explícitos de: los objetivos generales y específicos.

Posteriormente relacionar el cumplimiento de los objetivos específicos con tareas o actividades a desarrollar (al final se debe incluir seguramente actividades de validación y prueba del producto - plan de prueba).

\subsection{Planificación}
\lipsum[1]%reemplazar esta linea



\subsection{Equipo de Trabajo}
\lipsum[1]%reemplazar esta linea


\begin{landscape}
\subsection{Carta Gantt}\label{sec:gantt}
\begin{figure}[hbt]
  \centering
  %\includegraphics{}
  \caption{Carta Gantt del Proyecto XYZ}
  \label{gantt}
\end{figure}
\end{landscape}







En la Tabla \ref{t:info} se muestran las características de los sistemas GNU/Linux, obtenidas desde \cite{001}.


\begin{table}[hbt]
\begin{center}
\begin{tabular}{|l|p{10cm}|}\hline
\multicolumn{2}{|c|}{\textbf{Información general}}\\
\hline
\textbf{Modelo de desarrollo}&desarrollo	Software libre y código abierto\\
\textbf{Última versión estable}&Kernel: 4.11.3 (info) 25 de mayo de 2017 (10 días)\\
\textbf{Última versión en pruebas}&	4.12.rc2 (info) 22 de mayo de 2017 (13 días)\\
\textbf{Escrito en}&	C\\
\textbf{Núcleo}&	Núcleo Linux\\
\textbf{Plataformas soportadas}	& DEC Alpha, ARM, AVR32, Blackfin, ETRAX CRIS, FR-V, H8/300, Itanium, M32R, m68k, Microblaze, MIPS, MN103, PA-RISC, PowerPC, s390, S+core, SuperH, SPARC, TILE64, Unicore32, x86, Xtensa\\
\textbf{Licencia}	&GNU General Public License y otras\\
\textbf{Estado actual}	&En desarrollo\\
\textbf{En español}	&Sí\\
\hline
\end{tabular}
\end{center}
\caption{Información General de GNU/Linux}
\label{t:info}
\end{table}


\chapter{Desarrollo del Proyecto}
\lipsum[1]%reemplazar esta linea

\section{Definición del Problema}
\lipsum[1]%reemplazar esta linea

\section{Propuesta de Solución}
\lipsum[1]%reemplazar esta linea



\chapter{Conclusión}
En las conclusiones se destaca lo mostrado en el trabajo, resaltando los resultados. Se indican los trabajos futuros. Usualmente, luego de las conclusiones se incluye un párrafo de agradecimientos a quienes auspician la investigación.
\section{Principales aportes}
\lipsum[1]%reemplazar esta linea

\section{Contraste de resultados}
\lipsum[1]%reemplazar esta linea

\section{Trabajo Futuro}
\lipsum[1]%reemplazar esta linea





%%%%%
%agregar referencias
\bibliographystyle{IEEEtran}
%\nocite{*} % mostrar todas las referencias aunque no esten citadas
\bibliography{bibliografia.bib}

\renewcommand{\appendixname}{Anexos}
\appendix

\chapter{Definciones, Acronimos y Abreviaturas}\label{definiciones}
Funciona como un glosario para que cualquier lector pueda entender la terminología específica, los acrónimos (ej. UML, API) y las abreviaturas usadas en el informe.
\section{Definiciones}
\section{Acrónimos}
\section{Abreviaturas}

\chapter{Configuraciones}\label{configuracion}









\end{document}
