\documentclass[12pt]{report}
\def\TITULO{Título}
\def\subtitulo{Práctica ???}
\def\autora{Autor}
\def\correoa{correo@alumnos.ulagos.cl}
\def\asignatura{Práctica ???}
\def\campus{Campus ???}
\def\carrera{Ingeniería Civil en Informática}
\input{Macros}
%LENGUAJE DE PROGRAMACIÓN POR DEFECTO
\lstset{language=Python}


\begin{document}

	%%%%%%%%%%%PORTADA%%%%%%%%%%%%%%%%%%%%%
%\setlength{\unitlength}{1 cm} %Especificar unidad de trabajo
\maketitle

\cleardoublepage
\pagenumbering{roman}
\setcounter{page}{1}

%INDICE GENERAL
\tableofcontents
%INDICE DE FIGURAS
\listoffigures
%INDICE DE TABLAS
\renewcommand{\listtablename}{Índice de tablas}\listoftables
%%%%%%%%%%%%%FIN PORTADA%%%%%%%%%%%%%%%%
\renewcommand{\lstlistlistingname}{Índice de algoritmos}
\lstlistoflistings
%\addcontentsline

%\thispagestyle{empty}
%\begin{abstract}
%\end{abstract}
\cleardoublepage
\pagenumbering{arabic}
\setcounter{page}{1}
\chapter{Fundamentación}
\section{Introducción}
\instruccion{Escriba una introducción general que contextualice su práctica. Debe incluir:
\begin{itemize}
    \item Contexto: ¿Dónde y cuándo se realizó la práctica?
    \item Motivación: ¿Por qué eligió este centro de práctica?
    \item Resumen breve: ¿En qué consistió el trabajo a grandes rasgos?
    \item Estructura del informe: Describa brevemente qué se encontrará en los capítulos siguientes.
\end{itemize}}

Lorem ipsum dolor sit amet, consectetur adipiscing elit...

\section{Objetivos de la Práctica (Académicos)}
\instruccion{Esta sección es estática y define lo que la Universidad espera de usted según el Reglamento. \textbf{NO BORRE NI MODIFIQUE EL TEXTO NEGRO}, solo seleccione la opción (Intermedia o Profesional) que corresponda a su caso y borre la otra.}

\subsection{Objetivo General}

% ELIMINAR LA OPCIÓN QUE NO CORRESPONDA
\textbf{[OPCIÓN A: Práctica Intermedia (PP1)]}\\
Permitir que la o el estudiante se familiarice con el ámbito laboral, desarrollando las actividades en contacto constante con el centro de práctica.

\textbf{[OPCIÓN B: Práctica Profesional (PP2)]}\\
Permitir que la o el estudiante desempeñe labores relevantes que una o un Ingeniera/o Civil en Informática realiza en una organización.

\subsection{Objetivos Específicos}

% ELIMINAR LA OPCIÓN QUE NO CORRESPONDA
\textbf{[OPCIÓN A: Práctica Intermedia (PP1)]}
\begin{enumerate}\justifying
    \item Conocer la estructura de la organización y formas de operación de las áreas que componen el Centro de Práctica.
    \item Conocer y aprender métodos de trabajo y formas de comunicación utilizados en entornos multidisciplinarios y/o multiculturales.
    \item Visualizar y proponer el método de ingeniería para concebir, diseñar, implementar y operar actividades colaborativas en la solución de problemas, en áreas informáticas o afines a esta.
\end{enumerate}

\textbf{[OPCIÓN B: Práctica Profesional (PP2)]}
\begin{enumerate}\justifying
    \item Conocer la estructura de la organización y formas de operación de cada una de las áreas que componen el Centro de Práctica.
    \item Insertarse en grupos multidisciplinarios y/o multiculturales, participando activamente en equipos de trabajo, con capacidad de adaptación y pensamiento crítico.
    \item Aprender métodos de trabajo, tendientes a la operación, administración y gestión en áreas informáticas.
    \item Concebir, diseñar, implementar y/o operar soluciones a problemas propuestos o detectados desde el enfoque de la Ingeniería Civil en Informática.
\end{enumerate}

\section{Objetivos del Trabajo de Práctica (Del Proyecto)}
\instruccion{Aquí debe definir los objetivos técnicos de las tareas que su supervisor le asignó. Deben ser medibles y alcanzables.}

\subsection{Objetivo General del Trabajo}
\instruccion{Ejemplo: Desarrollar un módulo de facturación electrónica para el sistema ERP de la empresa para optimizar los tiempos de emisión de documentos.}

\subsection{Objetivos Específicos del Trabajo}
\begin{enumerate}\justifying
    \item Levantar requerimientos funcionales con el departamento de contabilidad.
    \item Diseñar la base de datos para el almacenamiento de DTEs.
    \item Implementar la API de conexión con el Servicio de Impuestos Internos.
    \item Realizar pruebas unitarias y de integración del módulo desarrollado.
\end{enumerate}

\section{Datos del Alumno}
\begin{description}\justifying
  \item [Nombre del alumno] Nombre completo
  \item [Año de ingreso] 20XX
  \item [Campus] Osorno / Puerto Montt
  \item [Tipo de práctica] Intermedia / Profesional
  \item [Fecha de realización] DD de MM de 20XX - DD de MM de 20XX
  \item [Total de horas] (Mínimo 162 hrs para PP1, 360 hrs para PP2)
\end{description}

\chapter{Empresa y Contexto Laboral}

\section{Identificación de la Empresa}
\begin{description}\justifying
  \item [Razón Social] Nombre legal
  \item [Nombre de Fantasía] Nombre comercial
  \item [RUT] XX.XXX.XXX-X
  \item [Rubro] Giro de la empresa
  \item [Dirección] Dirección completa
  \item [Sitio Web] \url{www.empresa.cl}
\end{description}

\section{Datos del Supervisor}
\instruccion{Esta información es vital para la validación de su práctica.}
\begin{description}\justifying
  \item [Nombre] Nombre del supervisor
  \item [Cargo] Cargo que ocupa
  \item [Correo electrónico] email@empresa.com
  \item [Teléfono] +56 9 XXXX XXXX
\end{description}

\section{Descripción de la Organización}
\instruccion{Describa la empresa más allá de su "misión y visión". Enfóquese en:
\begin{itemize}
    \item ¿Qué problema resuelve la empresa en el mercado?
    \item Tamaño (número de empleados, sucursales).
    \item Cultura organizacional (¿Es jerárquica? ¿Es ágil?).
\end{itemize}
}

\subsection{Organigrama General}
\instruccion{Inserte el organigrama de la empresa. Si es muy grande, simplifíquelo pero asegúrese de marcar dónde se encuentra el departamento de informática. Si no existe debe crearlo.}
\begin{figure}[H]
\centering
%\includegraphics[width=0.8\textwidth]{figures/organigrama_empresa}
\caption{Organigrama General de la Empresa}
\label{fig:organigrama}
\end{figure}

\section{Descripción del Área de Trabajo}
\instruccion{Profundice en el departamento donde usted trabajó (Gerencia de Sistemas, Área de Desarrollo, Soporte, etc.). Describa:
\begin{itemize}
    \item Estructura interna del equipo.
    \item Metodologías utilizadas por el equipo (Scrum, Kanban, Tradicional).
    \item Tecnologías que el equipo utiliza habitualmente.
\end{itemize}
}

\chapter{Desarrollo de la Práctica}

\section{Definición del Problema u Oportunidad}
\instruccion{Describa la necesidad que originó su trabajo. ¿Qué estaba "mal" o qué faltaba antes de que usted llegara? Esto demuestra su capacidad de detectar problemas (Art. 4, punto d).}

\section{Marco Tecnológico (Stack Técnico)}
\instruccion{Liste y describa brevemente las herramientas utilizadas, justificando su elección o imposición por la empresa.}
\begin{itemize}
    \item \textbf{Lenguajes:} Python, Java, etc.
    \item \textbf{Frameworks:} React, Django, Spring Boot.
    \item \textbf{Herramientas de Gestión:} Jira, Trello, Git.
    \item \textbf{Infraestructura:} AWS, Azure, Docker.
\end{itemize}

\section{Planificación (Plan de Trabajo)}
\instruccion{Presente cómo organizó su tiempo. Se recomienda encarecidamente incluir una Carta Gantt comparando lo Planificado vs. lo Realizado.}

\section{Descripción Detallada de Actividades}
\instruccion{Esta es la sección principal. No haga solo un listado tipo bitácora (Día 1: Hice esto). Agrupe por hitos o proyectos relevantes.
Para cada actividad relevante describa:
\begin{enumerate}
    \item \textbf{El desafío:} ¿Qué debía resolver?
    \item \textbf{La solución:} ¿Cómo aplicó la ingeniería para resolverlo? (Diseño, Algoritmos, Lógica).
    \item \textbf{La implementación:} Fragmentos de código clave, diagramas de arquitectura, capturas de pantalla.
\end{enumerate}
}

% Ejemplo de inclusión de código
%\begin{lstlisting}[caption={Algoritmo de conexión a la API}, label={lst:api}]
% def connect_to_api(url):
%     response = requests.get(url)
%     return response.json()
%\end{lstlisting}

\section{Resultados Obtenidos}
\instruccion{Muestre el producto final. ¿Funciona? ¿Se implementó? Muestre gráficos de rendimiento, interfaces finales o métricas de éxito (ej: "Se redujo el tiempo de carga en un 20\%").}



\chapter{Discusión y Conclusiones}

\section{Análisis Crítico y Discusión}
\instruccion{Reflexione sobre su desempeño.
\begin{itemize}
    \item ¿Logró aplicar las competencias de la carrera?.
    \item ¿Cómo fue la integración con el equipo multidisciplinario?
    \item Comparación teoría (universidad) vs. práctica (empresa).
\end{itemize}
}

\section{Detección de Mejoras y Aportes}
\instruccion{Basado en su observación, proponga mejoras para la empresa (técnicas o de gestión) y mejoras para la carrera (malla curricular). Esto es parte del objetivo de proponer soluciones con pensamiento crítico.}

\section{Conclusión General}
\instruccion{Cierre el informe resumiendo el impacto de la práctica en su formación profesional.}
%%%%%
\renewcommand{\refname}{Referencias}

%agregar referencias
\bibliographystyle{IEEEtran}
%\nocite{*} % mostrar todas las referencias aunque no esten citadas
\bibliography{bibliografia}

%\input{Ejemplos}%comentar/eliminar esta línea
\end{document}
