\documentclass[a0,portrait]{a0poster}

%%%%%%%%%%%%%%%%%%%%%%%%%%%%%%%%%%%%%%%%%
% a0poster Portrait Poster
% LaTeX Template
% Version 1.0 (22/06/13)
%
% The a0poster class was created by:
% Gerlinde Kettl and Matthias Weiser (tex@kettl.de)
% 
% This template has been downloaded from:
% http://www.LaTeXTemplates.com
%
% License:
% CC BY-NC-SA 3.0 (http://creativecommons.org/licenses/by-nc-sa/3.0/)
%
%%%%%%%%%%%%%%%%%%%%%%%%%%%%%%%%%%%%%%%%%

%----------------------------------------------------------------------------------------
%	PACKAGES AND OTHER DOCUMENT CONFIGURATIONS
%----------------------------------------------------------------------------------------


%\usepackage[papersize={1000mm,1500mm},tmargin=70mm,bmargin=70mm,lmargin=70mm,rmargin=30mm]{geometry}%comentar esta linea si desean usar solo tamaño A0, esto es para tamaño 150x100
\usepackage[utf8]{inputenc}
\usepackage[T1]{fontenc}
\usepackage[spanish,activeacute,es-tabla]{babel}
\usepackage{multicol} % This is so we can have multiple columns of text side-by-side
\columnsep=100pt % This is the amount of white space between the columns in the poster
\columnseprule=3pt % This is the thickness of the black line between the columns in the poster

\usepackage[svgnames]{xcolor} % Specify colors by their 'svgnames', for a full list of all colors available see here: http://www.latextemplates.com/svgnames-colors
\usepackage{mwe}
\usepackage[absolute]{textpos}
\usepackage[default]{roboto}
%\usepackage{times} % Use the times font
%\usepackage{palatino} % Uncomment to use the Palatino font
\usepackage{graphicx} % Required for including images
\graphicspath{{images/}} % Location of the graphics files
\usepackage{booktabs} % Top and bottom rules for table
\usepackage[font=small,labelfont=bf]{caption} % Required for specifying captions to tables and figures
\usepackage{cite,cancel,fancyvrb,textcomp,times,booktabs,amssymb,amsmath,ragged2e,float,subfig,xspace,epic,eepic,multicol,multirow,colortbl,color,url,hyperref,pict2e,array,listings,pgfpages,eurosym,wasysym,textcase,datetime,amsthm,amsfonts} % For math fonts, symbols and environments
%\usepackage{multicol,graphicx,fancyhdr,eso-pic,url,float,lmodern,listings,times,textcomp, amsthm,amsmath,amssymb,dsfont,color,colortbl,sidecap,xspace,epic,eepic,anysize,setspace, hyperref, multirow,algorithm,algpseudocode,enumitem,pdflscape,lscape,subfigure,csquotes}
\usepackage{wrapfig} % Allows wrapping text around tables and figures
%Colores Ulagos
%COLOREAR SISTEMA
\definecolor{gray97}{gray}{.97}
\definecolor{gray75}{gray}{.75}
\definecolor{gray45}{gray}{.45}
\definecolor{listinggray}{gray}{0.9}
\definecolor{lbcolor}{rgb}{0.9,0.9,0.9}
\definecolor{amarillo}{RGB}{255,183,27}
\definecolor{amarilloc}{RGB}{250,223,141}
\definecolor{verde}{RGB}{118,188,33}
\definecolor{verdec}{RGB}{172,219,144}
\definecolor{rojo}{RGB}{202,54,37}
\definecolor{rojoc}{RGB}{255,176,192}
\definecolor{azul}{RGB}{0,61,166}
\definecolor{celeste}{RGB}{143,199,232}
\definecolor{negro}{RGB}{35,31,32}
\definecolor{naranjo}{RGB}{255,103,29}
\definecolor{naranjoc}{RGB}{255,164,136}
\definecolor{morado}{RGB}{126,87,197}
\definecolor{moradoc}{RGB}{220,168,226}
\definecolor{gris}{RGB}{183,177,169}
\definecolor{grisc}{RGB}{216,209,202}
%COLOREAR TEXTO
\newcommand\rojo[1]{\textcolor[RGB]{202,54,37}{#1}}
\newcommand\rojoc[1]{\textcolor[RGB]{255,176,192}{#1}}
\newcommand\gris[1]{\textcolor[RGB]{183,177,169}{#1}}
\newcommand\grisc[1]{\textcolor[RGB]{216,209,202}{#1}}
\newcommand\azul[1]{\textcolor[RGB]{0,61,166}{#1}}
\newcommand\celeste[1]{\textcolor[RGB]{143,199,232}{#1}}
\newcommand\verde[1]{\textcolor[RGB]{118,188,33}{#1}}
\newcommand\verdec[1]{\textcolor[RGB]{172,219,144}{#1}}
\newcommand\naranjo[1]{\textcolor[RGB]{255,103,29}{#1}}
\newcommand\naranjoc[1]{\textcolor[RGB]{255,164,136}{#1}}
\newcommand\amarillo[1]{\textcolor[RGB]{255,183,27}{#1}}
\newcommand\amarilloc[1]{\textcolor[RGB]{250,223,141}{#1}}
\newcommand\morado[1]{\textcolor[RGB]{126,87,197}{#1}}
\newcommand\moradoc[1]{\textcolor[RGB]{220,168,226}{#1}}
\newcommand\negro[1]{\textcolor[RGB]{35,31,32}{#1}}

\providecommand{\keywords}[1]{\textbf{\textit{Palabras Clave---}} #1}
\newtheorem{ejemplo}{Ejemplo}
\newtheorem{solucion}{Solución}
\newtheorem{definir}{Definición}
\newtheorem{prueba}{Prueba}
\newtheorem{demo}{Demostración}
\newtheorem{obs}{Observación}



%%%CODIGOS DE PROGRAMACION
\lstset{%backgroundcolor=\color{lbcolor},
	frame=Ltb, framerule=0pt, aboveskip=0.5cm, tabsize=4, rulecolor=, %%%CAMBIAR POR LENGUAJE DE PREFERENCIA
	stringstyle=\ttfamily,  %basicstyle=\footnotesize,
	upquote=true, aboveskip={1.5\baselineskip}, columns=fixed, showstringspaces=false, extendedchars=true,breaklines=true, prebreak = \raisebox{0ex}[0ex][0ex]{\ensuremath{\hookleftarrow}}, showtabs=false, showspaces=false, showstringspaces=false,
	%tipos de letra y colores
	identifierstyle=\ttfamily,
	keywordstyle=\bfseries  \color[RGB]{0,2,216}, %palabras reservadas
	commentstyle= \scriptsize\color[rgb]{0,.5,0.2}, %comentarios
	stringstyle=\color{rojo},%cadena de texto
	%numeracion de lineas
	framexleftmargin=0.1cm,%framextopmargin=1pt, framexbottommargin=1pt,
	aboveskip=2.8mm,belowskip=-1mm,
	framesep=0pt, rulesep=.4pt, rulesepcolor=\color{gray75}, numbers=left, numbersep=15pt, numberstyle=\tiny, numberfirstline = false, breaklines=true,literate={á}{{\'a}}1 {é}{{\'e}}1 {í}{{\'i}}1 {ó}{{\'o}}1 {ú}{{\'u}}1
	{Á}{{\'A}}1 {É}{{\'E}}1 {Í}{{\'I}}1 {Ó}{{\'O}}1 {Ú}{{\'U}}1
	{à}{{\`a}}1 {è}{{\`e}}1 {ì}{{\`i}}1 {ò}{{\`o}}1 {ù}{{\`u}}1
	{À}{{\`A}}1 {È}{{\'E}}1 {Ì}{{\`I}}1 {Ò}{{\`O}}1 {Ù}{{\`U}}1
	{ä}{{\"a}}1 {ë}{{\"e}}1 {ï}{{\"i}}1 {ö}{{\"o}}1 {ü}{{\"u}}1
	{Ä}{{\"A}}1 {Ë}{{\"E}}1 {Ï}{{\"I}}1 {Ö}{{\"O}}1 {Ü}{{\"U}}1
	{â}{{\^a}}1 {ê}{{\^e}}1 {î}{{\^i}}1 {ô}{{\^o}}1 {û}{{\^u}}1
	{Â}{{\^A}}1 {Ê}{{\^E}}1 {Î}{{\^I}}1 {Ô}{{\^O}}1 {Û}{{\^U}}1
	{œ}{{\oe}}1 {Œ}{{\OE}}1 {æ}{{\ae}}1 {Æ}{{\AE}}1 {ß}{{\ss}}1
	{ű}{{\H{u}}}1 {Ű}{{\H{U}}}1 {ő}{{\H{o}}}1 {Ő}{{\H{O}}}1
	{ç}{{\c c}}1 {Ç}{{\c C}}1 {ø}{{\o}}1 {å}{{\r a}}1 {Å}{{\r A}}1
	{€}{{\EUR}}1 {£}{{\pounds}}1 {Ñ}{{\~N}}1 {ñ}{{\~n}}1 {¿}{{?`}}1
}
\renewcommand{\lstlistingname}{Código}

\def\TITULO{Titulo}
\def\autora{Autor}
\def\correoa{correo@alumnos.ulagos.cl}
\def\autorb{Autor}
\def\correob{correo@alumnos.ulagos.cl}
\def\autorc{Autor}
\def\correoc{correo@alumnos.ulagos.cl}
\def\autord{Autor}
\def\correod{correo@alumnos.ulagos.cl}
\def\autore{Autor}
\def\correoe{correo@alumnos.ulagos.cl}
\def\autorf{Autor}
\def\correof{correo@alumnos.ulagos.cl}
\def\autorg{Autor}
\def\correog{correo@alumnos.ulagos.cl}
\def\campus{Ciudad}
\def\carrera{Ingenier\'ia Civil en ...}
\lstset{language=PYTHON}%Cambiar el Lenguaje de Programación
%SQL, PYTHON, HTML5, PHP, JAVA, C
\def\proyecto{Feria de las Ingenierías}

\begin{document}
\begin{minipage}[t]{0.25\linewidth}
\includegraphics[width=\linewidth]{images/Logo-ULagos.png}
\end{minipage} \hfill
\begin{minipage}[c]{0.65\linewidth}
\veryHuge \color{azul} \scshape\textbf{\TITULO} \color{negro}\\ 
\huge\scshape\textit{\celeste{\carrera}}\\

    \huge \textbf{\autora - \autorb - \autorc \\ \autord - \autore - \autorf}\\[0.5cm] % Author(s)
\end{minipage}


\vspace{1cm} % A bit of extra whitespace between the header and poster content


\begin{multicols}{3} % Aca pueden cambiar cuantas columnas quieren el poster, 2 a 3 

%%RESUMEN
\begin{abstract}
Un resumen Establece el problema, Dice porqué es interesante, Señala los logros y desafíos; debe ser llamativo, motivador, descriptivo y sin contenido específico. \textbf{No incluye}: citas, referencias, conclusiones, figuras ni tablas.
\end{abstract}

%----------------------------------------------------------------------------------------
%	INTRODUCCION
%----------------------------------------------------------------------------------------

%\color{rojo} % Color de la introduccion
\color{negro}
\section*{Generalidades}

Es lo que comúnmente se conoce como introducción, conduce al lector desde un tema de un área general hacia un campo de investigación específico, describe el contexto, el problema, motiva al lector.

Introduce la terminología, destaca las contribuciones del documento y da una breve descripción de la organización de éste.


%----------------------------------------------------------------------------------------
%	OBJECTIVOS
%----------------------------------------------------------------------------------------

 %Color del contenido

\subsection*{Origen del Tema}
Contextualiza el trabajo respecto de investigaciones previas de otros autores y propias, señala las diferencias con trabajos previos. Algunas veces se incluye en la introducción o bien en la discusión del trabajo (secciones finales) \cite{000}. 

\subsection*{Justificación y Aporte}
Justificar la conveniencia del proyecto desde diversos puntos de vista \cite{003}.

Preguntas clave:
  \begin{itemize}
  \item ¿Para qué sirve la investigación?
  \item ¿Quiénes se benefician con los resultados?
  \item ¿Ayuda a resolver algún problema práctico?
  \item ¿Contribuye a aumentar el conocimiento?
  \item ¿Se podrán generalizar los resultado?
\end{itemize}

\section*{Objetivo General}
Optimizar e implementar soluciones algorítmicas eficientes, utilizando diversas estructuras de datos y distintos paradigmas de programación, para procesar reparaciones a bases de datos inconsistentes, garantizando la semántica de los datos con el propósito de obtener respuestas a consultas de agregación para la toma de decisiones.
\subsection*{Específicos}
\begin{enumerate}
 \item Identificar los operadores ideales para computar consultas en una base de datos.
\item Comparar las diversas estructuras de datos y lenguajes de programación que permitan resolver este problema eficientemente con su posterior implementación.
\item Verificar empíricamente que las soluciones algorítmicas permiten resolver el problema planteado, determinando su eficiencia al compararlos con diferentes herramientas existentes en la literatura.
\end{enumerate}

%----------------------------------------------------------------------------------------
%	MATERIALS AND METHODS
%----------------------------------------------------------------------------------------


\section*{Desarrollo}

En esta parte deben describir el trabajo realizado, desde el inicio hasta el final, incluyendo la descripción de las tablas, figuras, códigos y formulas usadas en el proyecto que se presentan en este informe, también si se usan referencias bibliograficas, deben incluirlas en el texto.

El Código \ref{codigo1} es... La Figura \ref{foto1} muestra .... la Tabla \ref{tabla1} presenta... El autor \cite{001} presenta. La ecuación \ref{ecuacion1} representa...

%------------------------------------------------

\subsection*{subtitulo}
\begin{lstlisting}[caption=Código de Ciclo while,label=codigo1]
i=0
p=0
n=0
while i<20:
    n=int(input('ingrese'))
    if n<0:
        n=n+1
    elif n>0:
        p=p+1
    else:
        print('hola')
    i=i+1
\end{lstlisting}


\begin{eqnarray}\label{ecuacion1}
\cos\bar{\phi}_j Q_{j+1,k,t} + Q_{j+1,k,y}+\frac{\sin^2\bar{\phi}_j}{T\cos\bar{\phi}_j} Q_{j+1,k}&=&\nonumber \\
-\cos\phi_j Q_{j,k,t} + Q_{j,k,y}-\frac{\sin^2\phi_j}{T\cos\phi_j} Q_{j,k}.\label{edgej}
\end{eqnarray} 

%----------------------------------------------------------------------------------------
%	RESULTS 
%----------------------------------------------------------------------------------------
\subsection*{Resultados}
En este apartado deben destacar lo logrado en el proyecto.


\begin{center}\vspace{1cm}
\includegraphics[width=0.8\linewidth]{placeholder}
\captionof{figure}{Forma de poner imágenes}
\label{foto1}
\end{center}\vspace{1cm}

\begin{center}\vspace{1cm}
    \begin{tabular}{ l l p{8cm}}\hline
       texto1  & texto2 & texto3\\\hline
        texto4 texto4 & texto5 texto5 texto5& Es lo que comúnmente se conoce como introducción, conduce al lector desde un tema de un área general hacia un campo de investigación específico, describe el contexto, el problema, motiva al lector.\\\hline
        texto7 &texto8  &texto9 \\\hline
    \end{tabular}
\captionof{table}{Forma de poner Tablas}
\label{tabla1}
\end{center}\vspace{1cm}


%----------------------------------------------------------------------------------------
%	CONCLUSIONS
%----------------------------------------------------------------------------------------

%\color{rojo} % SaddleBrown color for the conclusions to make them stand out

\section*{Conclusiones}

En las conclusiones se destaca lo mostrado en el trabajo, resaltando los resultados. Se indican los trabajos futuros.

\subsection*{Trabajo Futuro}

Deben explicar que cosas del proyecto podrían mejorar a futuro, que cosas se podrían añadir, etc.

 %----------------------------------------------------------------------------------------
%	REFERENCES
%----------------------------------------------------------------------------------------
\color{celeste}
%\nocite{*} % mostrar todas las referencias aunque no esten citadas
\bibliographystyle{IEEEtran} % Plain referencing style
\bibliography{bibliografia.bib} % Use the example bibliography file sample.bib

%----------------------------------------------------------------------------------------
%	ACKNOWLEDGEMENTS
%----------------------------------------------------------------------------------------

%\section*{Agradecimientos}

%Etiam fermentum, arcu ut gravida fringilla, dolor arcu laoreet justo, ut imperdiet urna arcu a arcu. Donec nec ante a dui tempus consectetur. Cras nisi turpis, dapibus sit amet mattis sed, laoreet.

%----------------------------------------------------------------------------------------

\end{multicols}
%%%%NO TOCAR%%%%%%%
\begin{textblock*}{\paperwidth}(50mm,\paperheight)% Lower
  \raggedright% left edge of page
  \raisebox{0pt}[0pt][0pt]{\includegraphics[width=20cm]{images/Logo-Acreditacion.png}}
\end{textblock*}
\begin{textblock*}{\paperwidth}(30mm,\paperheight)% Lower
  \centering% left edge of page
  \raisebox{40pt}[0pt][0pt]{\Huge \azul{\proyecto, \campus}}
\end{textblock*}
\begin{textblock*}{\paperwidth}(-10mm,\paperheight)% Lower
  \raggedleft% right edge of page
  \raisebox{0pt}[0pt][0pt]{\includegraphics[width=10cm]{images/web.png}}
  \end{textblock*}
\end{document}