\chapter{Metodología de Trabajo}\label{fundamentacion}
Detalla el plan y la estrategia que se seguirán para desarrollar el proyecto, desde su justificación hasta la asignación de recursos.



\section{Alcance}
Define con precisión los límites del proyecto. Debe quedar claro qué hará y qué no hará el software.
\begin{itemize}
    \item Identificar el producto por su nombre. 
    \item Describir la aplicación, sus beneficios y objetivos.
\end{itemize}

Esta subdivisión debe:
\begin{enumerate}\justifying
  \item Identifique el producto del software para ser diseñado por el nombre (por ejemplo, Anfitrión DBMS, el Generador del Reporte, etc.);
  \item Explique eso que el producto (del software hará y que no hará.
  \item Describe la aplicación del software especificándose los beneficios pertinentes, objetivos, y metas;
  \item Sea consistente con las declaraciones similares en las especificaciones de niveles superiores (por ejemplo, las especificaciones de los requisitos del sistema), si ellos existen.
\end{enumerate}




\section{Objetivos}\label{objetivos}
Expresan los fines concretos del proyecto. Deben ser medibles, factibles y comenzar con un verbo en infinitivo (ej. Analizar, Diseñar, Implementar)

\subsection{Objetivo General}
Declara el propósito principal y final del proyecto en una sola frase.

\begin{ejemplo}
Desarrollar e implementar un sistema de base de datos web integral y eficiente que satisfaga los requisitos de gestión de información académica y de investigación de [Nombre del Cliente o Empresa Cliente], permitiendo la administración centralizada y estructurada de datos relativos a profesores, alumnos, proyectos, departamentos y asignaturas, y facilitando el acceso y la operatividad a los usuarios designados.
\end{ejemplo}
\subsection{Objetivos Específicos}
Son los pasos o hitos necesarios para alcanzar el objetivo general.
\begin{ejemplo}
\begin{enumerate}
    \item \textbf{Formulación y Modelado Conceptual}:
   Analizar exhaustivamente los requisitos funcionales y de datos proporcionados por el cliente, para formular el presente informe técnico y diseñar un modelo de datos conceptual y lógico que represente fielmente las entidades, atributos y relaciones identificadas, sirviendo como base para la estructura de la base de datos.
   \item \textbf{Diseño e Implementación de la Base de Datos}:
   Traducir el modelo de datos lógico a un diseño físico de base de datos optimizado, seleccionando un sistema gestor de base de datos (SGBD) adecuado, y proceder con la creación de la estructura de la base de datos, incluyendo tablas, relaciones, restricciones de integridad y los índices necesarios para asegurar la eficiencia y consistencia de los datos.
   \item \textbf{Diseño y Desarrollo del Sistema Web con Integración de Datos}:
   Diseñar la arquitectura del sistema web y desarrollar una interfaz de usuario intuitiva y funcional que permita la interacción con la base de datos implementada. Esto incluye la implementación de los procesos CRUD (Crear, Leer, Actualizar, Eliminar) para todas las entidades gestionadas, asegurando una correcta integración entre la capa de presentación y la capa de datos.
   \item \textbf{Pruebas, Validación y Despliegue}:
   Ejecutar un plan de pruebas exhaustivo que abarque pruebas unitarias, de integración, de sistema y de aceptación del usuario, con el fin de validar el correcto funcionamiento del sistema web y la base de datos, asegurar el cumplimiento de los requisitos iniciales, corregir posibles errores y preparar el sistema para su despliegue en el entorno productivo del cliente.
\end{enumerate}
\end{ejemplo}

\section{Desglose de Actividades y Tareas}\label{s:actividades}
Por cada objetivo específico, se listan las actividades y tareas concretas que se realizarán para cumplirlo.

\begin{ejemplo}
\subsection{Objetivo Específico 1: \dots}
\subsubsection{Actividades y Tareas}
\begin{enumerate}
    \item Actividad 1:
    \begin{enumerate}
        \item Tarea 1:
        \item Tarea 2: 
    \end{enumerate}
    \item Actividad 2:
   \begin{enumerate}
        \item Tarea 1:
        \item Tarea 2: 
    \end{enumerate}
 
\end{enumerate}
\end{ejemplo}




\section[Planificación Temporal]{Planificación Temporal del Proyecto}
Presenta el cronograma del proyecto.
\begin{landscape}
\subsection{Carta Gantt}\label{sec:gantt}

Diagrama visual de las actividades en el tiempo. Si el proyecto es águil no se recomienda usar Gantt.

\begin{figure}[hbt]
  \centering
  %\includegraphics{}
  \caption{Carta Gantt del Proyecto XYZ}
  \label{gantt}
\end{figure}
\end{landscape}


\subsection{Desglose de Actividades}\label{s:PERT}
En esta sección presentan las actividades de la Sección \ref{s:actividades} con su duración, dependencias, caminos críticos, entre otras y se debe dar una conclusión de lo mismo.
\begin{figure}[hbt]
\begin{tabular}{|c|c|c|c|c|}\hline
  \textbf{Actividad}&\textbf{Duración} &\textbf{Después de} & \textbf{Simultanea} & \textbf{Antes de}\\\hline
& & &&\\\hline

\end{tabular}
  \caption{Duración de tareas y dependencias}
\end{figure}

\begin{landscape}
\begin{figure}[hbt]
  \centering
  %\includegraphics{}
  \caption{Grafo de Actividades del Proyecto XYZ}
  \label{CPM}
\end{figure}
\end{landscape}

\begin{landscape}
\begin{figure}[hbt]
  \centering
  %\includegraphics{}
  \caption{Grafo de Actividades con duración del Proyecto XYZ}
  \label{CPMduracion}
\end{figure}
\end{landscape}

\begin{figure}[hbt]
 \begin{tabular}{|c|c|cc|cc|c|c|}\hline
 & & \multicolumn{2}{|c|}{\textbf{Inicio}} & \multicolumn{2}{|c|}{\textbf{Termino}} & \textbf{Holgura} & \\
\textbf{Actividad}& \textbf{Duración}& \textbf{Temprano} &\textbf{Tardío} &\textbf{Temprano} &\textbf{Tardío} &\textbf{Total}  &\textbf{Crítico} \\\hline
& & &   & &   & & \\\hline

\end{tabular}
  \caption{Cálculo del diagrama de actividades}
\end{figure}


\section{Análisis de Viabilidad del Proyecto}
Evalúa si el proyecto puede llevarse a cabo con los recursos y plazos disponibles.


\subsection{Viabilidad Técnica}
  ¿Se cuenta con la tecnología y el conocimiento necesarios? 

  Los siguientes subtítulos son algunas ideas:
  
\subsubsection{Tecnologías Maduras y Disponibles}
\subsubsection{Complejidad Manejable}
\subsubsection{Recursos Humanos Calificados}
\subsubsection{Infraestructura Requerida}
\subsubsection{Escalabilidad y Rendimiento}
\subsection{Viabilidad Operativa}
¿El sistema será aceptado y utilizado por los usuarios?
\subsection{Viabilidad de Plazos (Cronograma)}
¿El cronograma propuesto es realista?

\section{Metodología de Desarrollo (\rojo{OBLIGATORIO})}
Aquí se debe explicar en detalle la metodología de desarrollo elegida (ej. Scrum, Kanban, XP, Cascada) y justificar por qué es la más adecuada para este proyecto en particular.