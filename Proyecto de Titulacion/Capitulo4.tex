
\chapter{Requisitos del Proyecto}\label{requisitos}
Especifica en detalle qué debe hacer el sistema. Es recomendable complementar esta sección con diagramas UML para mayor claridad.
\section{Requisitos de Interfaces Externos (\rojo{OPCIONAL})}
Detalla cómo el sistema interactuará con usuarios y otros sistemas.
\subsection{Interfaces de Usuario}
\subsection{Interfaces de Hardware}
\subsection{Interfaces de Software}
\subsection{Interfaces de Comunicación}

\section{Requisitos Funcionales (\rojo{Obligatorio})}
Describen las funciones que el sistema debe ser capaz de realizar.
\begin{ejemplo}
    \begin{itemize}
        \item \textbf{RF-01}: El sistema debe permitir al administrador registrar nuevos usuarios.
\item \textbf{RF-02}: El sistema debe generar un reporte de inscripciones en formato PDF.
    \end{itemize}
\end{ejemplo}
\subsection{Requisito funcional x}
\subsection{Requisito funcional y}
\section{Requisitos de Rendimiento (\rojo{Opcional})}
Especifica qué tan bien debe funcionar el sistema en términos de velocidad, capacidad de respuesta y uso de recursos.
\begin{ejemplo}
 El sistema debe cargar la página principal en menos de 2 segundos con 100 usuarios concurrentes.
\end{ejemplo}
\section{Requisitos de Desarrollo (\rojo{Opcional})}
 Impone restricciones sobre el proceso de construcción del sistema, como el uso de un estándar de codificación específico o la obligación de usar un sistema de control de versiones.
\section{Requisitos de Tecnológicos (\rojo{Obligatorio})}
Detalla las tecnologías específicas (lenguajes, frameworks, bases de datos) que se utilizarán.
\begin{ejemplo}
    \begin{itemize}
        \item \textbf{RT-01}: El backend se desarrollará en Python con el framework Django.
\item \textbf{RT-02}: La base de datos a utilizar será PostgreSQL.
    \end{itemize}
\end{ejemplo}
\section{Seguridad (\rojo{Obligatorio})}
Define los requisitos para garantizar la seguridad del sistema y sus datos.
\begin{ejemplo}
    \begin{itemize}
        \item \textbf{RS-01}: Todas las contraseñas de usuario deben almacenarse de forma cifrada
    \end{itemize}
\end{ejemplo}
\section{Modelo de  Datos (\rojo{Obligatorio})}
Describe la estructura de los datos. Se debe incluir un Modelo Entidad-Relación  y el Modelo Relacional normalizado.  Si se usan bases de datos no relacionales, se debe mostrar la estructura (ej. JSON).

\begin{landscape}
\begin{figure}[hbt]
  \centering
  %\includegraphics{}
  \caption{Modelo Entidad Relación}
  \label{f:MER}
\end{figure}
\end{landscape}

\begin{landscape}
\begin{figure}[hbt]
  \centering
  %\includegraphics{}
  \caption{Modelo Relación Relacional}
  \label{f:MR}
\end{figure}
\end{landscape}


