
\chapter{Aseguramiento de la Calidad (\rojo{Opcional})}\label{calidad}
Define el plan y las técnicas que se utilizarán para garantizar que el software cumpla con los estándares de calidad definidos. Incluye el modelo de calidad, las técnicas a aplicar y cómo se implementarán.
\section{Modelo}
\section{Técnica}
\section{Implementación}

\chapter{Proceso de Prueba del Software (\rojo{Opcional})}\label{prueba}
Describe la estrategia de pruebas para verificar que el software funciona correctamente y sin errores. Detalla los diferentes niveles de prueba que se aplicarán.
\section{Criterios de Prueba}
 Define los enfoques, como pruebas de Caja Negra (sin ver el código) o Caja Blanca (conociendo el código).
\subsection{Enfoque de Prueba de Caja Negra}
\subsection{Enfoque de Prueba de Caja Blanca}

\section{Prueba de Unidad}
Probar componentes individuales.
\section{Prueba de Integración}
 Probar cómo interactúan los componentes entre sí.
\subsection{Integración Descendente}
\subsection{Integración Ascendente}
\subsection{Prueba de Regresión}
\section{Prueba de Sistema}
Probar el sistema completo contra los requisitos.
\subsection{Prueba de Recuperación}
\subsection{Prueba de Seguridad}
\subsection{Prueba de Resistencia}
\subsection{Prueba de Rendimiento}
\section{Prueba de Aceptación}
Realizada por el usuario final para validar que el sistema cumple sus necesidades

