
\chapter{Análisis Económico}\label{economico}
\juaramir{Este capítulo es obligatorio para todo proyecto que implique desarrollo de software y no sea de investigación}
Este capítulo evalúa la viabilidad del proyecto desde una perspectiva de negocio y financiera. Es especialmente útil si el software tiene un potencial comercial.
\section{Estudio de Mercado}
Valida la idea del software como un producto viable en el mercado. Incluye un análisis de los posibles consumidores (demanda), la competencia (oferta y precios) y la estrategia de comercialización.

\subsection{Consumidores y análisis de la demanda}
 Identificar el público objetivo, estimar el tamaño del mercado potencial y proyectar la demanda futura del software. Esto es fundamental para justificar que el proyecto es necesario.
\subsection{Análisis de la oferta y los precios}
Investigar a la competencia (otras soluciones de software) y definir un modelo de negocio y precios (ej. licencia, suscripción, freemium)
\subsection{Comercialización}
Describir cómo se dará a conocer y se distribuirá el software a los clientes
\section{Evaluación de Proyectos de Inversión}
Utiliza indicadores financieros para medir la rentabilidad del proyecto.
\subsection{Criterios de evaluación}
Utilizar los flujos de caja para calcular indicadores clave de rentabilidad
\subsection{Valor Actual Neto (VAN)}
Determinar si el proyecto crea valor, es decir, si los beneficios futuros superan la inversión inicial, a una tasa de descuento definida.
\subsection{Tasa Interna de Retorno (TIR)}
 Calcular la rentabilidad intrínseca del proyecto para compararla con el costo de oportunidad del capital
\subsection{Análisis de riesgo e incertidumbre}
Plantear diferentes escenarios (optimista, pesimista) para evaluar la robustez financiera del proyecto de software ante cambios en la demanda o en los costos

\section{Viabilidad Económica}
Detalla todos los costos asociados al proyecto y los beneficios esperados.
\subsection{Costos de Desarrollo}
Inversión inicial en personal, hardware y software.
\subsubsection{Personal}
\subsubsection{Software}
Estimar la infraestructura necesaria (servidores, capacidad de base de datos) en función de la demanda de usuarios proyectada
\subsubsection{Hardware/Infraestructura}
Cuantificar la inversión inicial en hardware (computadores, servidores), software (licencias de sistemas operativos, IDEs) y otros activos
\subsection{Costos Operativos (Post-Implementación)}
Estimar los costos recurrentes como salarios del equipo de desarrollo y mantención, costos de hosting, dominios, soporte técnico, etc.
\begin{itemize}
    \item Mantenimiento del software y la base de datos.
\item Soporte técnico a usuarios.
\item Costos continuos de infraestructura (energía, conectividad, cloud si aplica).
\item Capacitación continua.
\end{itemize}
\subsection{Retorno de la Inversión (ROI) y Beneficios}
Cuantifica el retorno financiero y describe los beneficios cualitativos (ej. mejora de la eficiencia, reducción de errores). Aunque muchos beneficios son cualitativos, su impacto económico es considerable:
\subsubsection{Reducción de Costos Directos}
\subsubsection{Eficiencia Mejorada}
\subsubsection{Mejora en la Toma de Decisiones}
\subsubsection{Cumplimiento y Acreditación}
\subsubsection{Reducción de Errores}
Aquí hay que añadir secciones según materia vista en el curso de formulación y evaluación de proyectos.


deben añadir análisis económico de carta planificación temporal como la gantt (Figura \ref{gantt}), y el desglose de actividades de la Sección \ref{s:PERT} y malla CPM (Figura \ref{CPMcritico})



\begin{landscape}
\begin{figure}[hbt]
  \centering
  %\includegraphics{}
  \caption{Grafo de Actividades con duración y caminos críticos}
  \label{CPMcritico}
\end{figure}
\end{landscape}
