
\chapter{Formulación del Proyecto}\label{formulacion}
Esta sección es la introducción formal del proyecto. Debe guiar al lector desde el contexto general hasta el problema específico, justificando la necesidad del trabajo y presentando la solución propuesta.


Introduce la terminología, destaca las contribuciones del documento y da una breve descripción de la organización de éste


\section{Propósito}
Contextualiza el proyecto en relación con otros trabajos o investigaciones existentes. Señala qué lo hace diferente o cómo se basa en investigaciones previas. Largo aproximado: 2 páginas.

\begin{ejemplo}
    Mientras que existen múltiples sistemas de gestión académica, la mayoría son soluciones genéricas. Este proyecto se diferencia al proponer un sistema diseñado específicamente para los flujos de trabajo del Departamento de Ciencias de la Ingeniería de la Universidad de Los Lagos, integrando funcionalidades únicas de seguimiento de tesis que no se encuentran en alternativas comerciales.
\end{ejemplo}

\section{Identificación del Problema}
Define clara y detalladamente el problema que el proyecto busca resolver. Expone las deficiencias de los métodos actuales y por qué es necesario encontrar una nueva solución.
\section{Justificación y Aporte}
Argumenta por qué el proyecto es importante y conveniente. Responde a preguntas como: ¿Para qué sirve? , ¿quiénes se benefician? , ¿resuelve un problema práctico?


\subsection{Utilidad del Sistema}
Explica cómo el sistema mejorará procesos como la centralización de información  o la optimización de tareas. 

Los siguientes subtítulos son algunas ideas:
\subsubsection{Centralización e Integración de la Información}
\subsubsection{Optimización de Procesos}
\subsubsection{Facilitación del Acceso a la Información}
\subsubsection{Soporte para la Generación de Informes y Estadísticas}
\subsection{Aporte del Sistema a \dots...}
Detalla los beneficios institucionales, como la mejora en la toma de decisiones  o la modernización tecnológica.

Los siguientes subtítulos son algunas ideas:

\subsubsection{Mejora de la Eficiencia Operativa}
\subsubsection{Fortalecimiento de la Toma de Decisiones Estratégicas}
\subsubsection{Impulso a la Gestión}
\subsubsection{Mejora en la Gestión y Apoyo}
\subsubsection{Incremento de la Transparencia y Rendición de Cuentas}
\subsubsection{Modernización Tecnológica y Base para el Futuro}

\section{Marco Teórico}
Presenta la base teórica que sustenta el proyecto. Aquí se definen los conceptos, modelos y tecnologías fundamentales que se utilizarán para interpretar los resultados.  Ayuda a centrar la investigación y a prevenir errores.  Se construye a partir de una revisión bibliográfica de fuentes primarias, secundarias y terciarias.

\begin{itemize}\justifying
  \item \naranjo{Primarias (directas):} Libros, artículos, antologías, tesis, disertaciones, entre otros.
  \item \naranjo{Secundarias:} Compilaciones, resúmenes de listados de referencias publicadas en un área en particular, bases de datos.
  \item \naranjo{Terciarias:} Documentos que reúnen nombres y títulos de revistas y otras publicaciones.
\end{itemize}


 Ayuda a prevenir errores que se han cometido en otros estudios, conduce al establecimiento de la hipótesis o afirmaciones que se someterán a prueba.
 



\begin{ejemplo}
Para el desarrollo de este proyecto, se aplicará la metodología de desarrollo ágil Scrum, ya que permite una adaptación continua a los requisitos cambiantes. El sistema se fundamentará en una arquitectura de microservicios para garantizar la escalabilidad. La base de datos utilizará el modelo relacional, específicamente la Tercera Forma Normal (3FN), para asegurar la integridad y evitar la redundancia de datos...
\end{ejemplo} 

\section{Estado del Arte}\label{alternativas}
Investiga y describe las soluciones, herramientas o tecnologías existentes que ya intentan resolver el mismo problema o uno similar. Al final, se debe incluir una comparativa directa entre esas soluciones y la propuesta del proyecto, destacando las ventajas y novedades de esta última. 

\subsection{Comparativa propuesta con estado del arte}