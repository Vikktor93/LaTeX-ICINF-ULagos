\chapter{Descripción General del Sistema}\label{descripcion}
Ofrece una vista general del producto de software, sus funciones, usuarios y las condiciones bajo las cuales operará.

\section{Perspectiva del Producto}
Describe el producto en relación con otros sistemas y el entorno en el que funcionará.
\section{Funciones del Sistema}
Acá se describen las partes más relevantes que tendrá el sistema.
\section{Características de los Usuarios}
Define los diferentes tipos de usuarios que interactuarán con el sistema (ej. administrador, usuario final, etc.) y sus niveles de conocimiento técnico.
\section{Restricciones}
Enumera las limitaciones o restricciones que afectan el desarrollo, como el uso de un sistema operativo específico, políticas de la empresa o un presupuesto limitado.
\section{Suposiciones y Dependencias}
\subsection{Suposiciones}
Factores que se consideran verdaderos pero que no se pueden verificar al inicio del proyecto (ej. "Se asume que la universidad proporcionará acceso a la API del sistema de matrículas")
\subsection{Dependencias}
Factores externos que el proyecto necesita para poder completarse (ej. "El proyecto depende de la compra de una licencia para el motor de base de datos Oracle").
