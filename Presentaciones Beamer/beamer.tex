\documentclass[aspectratio=169]{beamer}%[handout]
\input{Macros}

\def\titulo{Título}
\def\subtitulo{Subtítulo}
\def\autora{Autor}
\def\correoa{correo@ulagos.cl}
\def\autorb{Autor}
\def\correob{correo@ulagos.cl}
\def\autorc{Autor}
\def\correoc{correo@ulagos.cl}
\def\asignatura{Asignatura}
\def\campus{Campus}
\def\carrera{Ingeniería Civil en Informática}

\lstset{language=SQL}

%Unico AUTOR
%\author[\autora]{\hspace{5cm}\raggedright\autora\\\hspace{5cm}\href{mailto:\correoa}{\correoa}}
   \author[NOMBRE AUTOR]{
	\href{mailto:\correoa}{\autora}\\
	\href{mailto:\correob}{\autorb}\\
	\href{mailto:\correoc}{\autorc}}

%%%%%%%%%%%%%FIN PORTADA


%%%%%Carpeta predeterminada fotos


\begin{document}


%Pagina de Portada
\begin {frame} [plain]

\titlepage
\end {frame}
\setbeamertemplate{background}{}
%%%%%%%%%%%%%%%%ÍNDICES
\section[Contenido]{}
\frame{
  \frametitle{\textbf{Contenido}}
\setcounter{tocdepth}{2}%1: solo titulo principal, 2: titulo y subtitulo, 3....
\scriptsize
\tableofcontents[]
}%Generacion de Indice por capitulo
\AtBeginSection[]{
\begin{frame}
\frametitle{\textbf{Contenido}}
\scriptsize
\tableofcontents[currentsection]
\end{frame}}

\AtBeginSubsection[]{
\begin{frame}
\frametitle{\textbf{Contenido}}
\scriptsize
\tableofcontents[currentsection,currentsubsection]
\end{frame}
}
%%%%%%%%%%%%%%%%FIN ÍNDICES







%%%%%%%%%%%%%%%%%%%%%%%%%%%%%%%%%%%%%
%%%%%%%%%%%%%%%%%INICIO PRESENTACION
\section{Introducción}

\begin{frame}[fragile]
	\frametitle{Texto con cita}
	\justifying
Este es un ejemplo del uso para citar a un solo autor \cite{000}, a dos autores \cite{000,001}
y a tres autores \cite{002, 003, 004}.

\end{frame}


\begin{frame}[fragile, allowframebreaks]
	\frametitle{Texto Dividido automaticamente con allowframebreaks}
	\justifying
En un lugar de la Mancha, de cuyo nombre no quiero acordarme, no ha mucho tiempo que vivía un hidalgo de los de lanza en astillero, adarga antigua, rocín flaco y galgo corredor. Una olla de algo más vaca que carnero, salpicón las más noches, duelos y quebrantos los sábados, lantejas los viernes, algún palomino de añadidura los domingos, consumían las tres partes de su hacienda.

El resto della concluían sayo de velarte, calzas de velludo para las fiestas con sus pantuflos de lo mesmo, los días de entre semana se honraba con su vellorí de lo más fino. Tenía en su casa una ama que pasaba de los cuarenta, y una sobrina que no llegaba a los veinte, y un mozo de campo y plaza, que así ensillaba el rocín como tomaba la podadera.

Frisaba la edad de nuestro hidalgo con los cincuenta años, era de complexión recia, seco de carnes, enjuto de rostro; gran madrugador y amigo de la caza. Querían decir que tenía el sobrenombre de Quijada o Quesada (que en esto hay alguna diferencia en los autores que deste caso escriben), aunque por conjeturas verosímiles se deja entender que se llamaba Quejana.

Pero esto importa poco a nuestro cuento; basta que en la narración dél no se salga un punto de la verdad. Es, pues, de saber, que este sobredicho hidalgo, los ratos que estaba ocioso (que eran los más del año) se daba a leer libros de caballerías con tanta afición y gusto, que olvidó casi de todo punto el ejercicio de la caza, y aun la administración de su hacienda.

Y llegó a tanto su curiosidad y desatino en esto, que vendió muchas hanegas de tierra de sembradura, para comprar libros de caballerías en que leer; y así llevó a su casa todos cuantos pudo haber dellos; y de todos ningunos le parecían tan bien como los que compuso el famoso Feliciano de Silva: porque la claridad de su prosa, y aquellas intrincadas razones suyas, le parecían de perlas.

\end{frame}

\section{Marco Teórico}
\section{Estado del Arte}
\section{Justificación}
\section{Objetivos}
\section{Desarrollo}
\subsection{Metodología de Desarrollo}
\subsection{Modelo de Datos}
\subsection{Requisitos}
\begin{frame}[fragile]
	\frametitle{Colores y Bloques}
	\justifying
	Lo siguiente son ejemplos de texto en colores: \rojo{Hola}, \verde{Hola}, \azul{hola}, \naranjo{hola}

	\begin{block}{Titulo}\justifying
		Contenido
	\end{block}


	\begin{exampleblock}{Titulo ejemplo}\justifying
		Contenido
	\end{exampleblock}


	\begin{alertblock}{Titulo alerta}\justifying
		Contenido
	\end{alertblock}

	\begin{sintaxis}\justifying
	\end{sintaxis}

	\begin{resultado}\justifying
	\end{resultado}

\end{frame}


\begin{frame}[fragile]
	\frametitle{Items, Descripciones y Enumeraciones}
	\justifying
	\transdissolve
	\begin{itemize}\justifying
		\item Item sin números
		\begin{itemize}\justifying
			\item nivel 2
			\begin{itemize}\justifying
				\item nivel 3
			\end{itemize}
		\end{itemize}
	\end{itemize}

	\begin{enumerate}\justifying
		\item Item Numerado
		\begin{enumerate}\justifying
			\item Nivel 2
			\begin{enumerate}\justifying
				\item Nivel 3
			\end{enumerate}
		\end{enumerate}
	\end{enumerate}

	\begin{description}\justifying
		\item[Descripción] Texto descrito
		\begin{description}\justifying
			\item[Nivel 2] Texto
		\end{description}

	\end{description}
\end{frame}




\subsection{Ejemplos}
\begin{frame}[fragile]
	\justifying
	\frametitle{Código}
	\lstset{language=C}
	\begin{lstlisting}
#include<stdio.h>
int main(){
	printf("Hola Mundo1");
	return 0;
}
	\end{lstlisting}
	\begin{block}{SQL}%\vspace{-0.3cm}
		\lstset{language=SQL}
		\begin{lstlisting}
Select *
from navegantes n
where categoria=7;
		\end{lstlisting}%\vspace{-0.3cm}
	\end{block}

\end{frame}

\begin{frame}[fragile]
	\frametitle{Código multicolumna}
	\justifying
	\lstset{language=C}
	\begin{minipage}[t]{0.49\textwidth}
		\begin{lstlisting}
#include<stdio.h>
int main(){
	printf("Hola Mundo2");
	return 0;
}
		\end{lstlisting}
	\end{minipage} \hfill
	\begin{minipage}[t]{0.49\textwidth}
		\lstset{firstnumber=6} %cambiar numero de inicio
		\begin{lstlisting}
#include<stdio.h>
int main(){
	printf("Hola Mundo3");
	return 0;
}
		\end{lstlisting}
	\end{minipage}
\end{frame}

\begin{frame}[fragile]
	\frametitle{Ejemplo de Función Matemática}
	\justifying
	\begin{displaymath}
		C_L=\frac{(S_{22}-\delta S_{11}^*)^*}{|\varPi S_{22}|^2=-|\pi|^2}
	\end{displaymath}

	\begin{displaymath}
		R_S=\frac{\sqrt{1-g_s}\cdot (1-|S_{11}|^2)}{1-(1-g_s)\cdot|S_{11}|^2}
	\end{displaymath}

\end{frame}



\begin{frame}[fragile]
	\frametitle{Tabla}
	\justifying
	\begin{center}
		\begin{tabular}{|c|c|c|c|c|c|}\hline
			\textbf{S}&\textbf{SCT} &\textbf{Asignatura}&\multicolumn{2}{|c|}{\textbf{Total Horas}}&\textbf{Previatura} \\\cline{4-5}
			&&&\textbf{TP}&\textbf{TA}&\\\hline
			\multirow{5}{*}&a&b&c&d&r\\\hline
		\end{tabular}
	\end{center}
\end{frame}


\begin{frame}[fragile]
	\frametitle{Diapositiva con avances}
	\justifying

	\begin{minipage}[t]{0.49\textwidth}
\only<1->{
\begin{block}{}
contenidos...
\end{block}
}
	\end{minipage} \hfill
	\begin{minipage}[t]{0.49\textwidth}
\only<2->{
\begin{exampleblock}{}
contenidos...
\end{exampleblock}
}
	\end{minipage}
\only<3>{
\begin{alertblock}{}
contenidos...
\end{alertblock}
}
\end{frame}


\section{Conclusión y Trabajo Futuro}
\begin{frame}[fragile]
	\frametitle{Conclusión}
	\justifying

\end{frame}

\begin{frame}[fragile]
	\frametitle{Trabajo Futuro}
	\justifying

\end{frame}


\begin{frame}[allowframebreaks]
	\frametitle{Referencias}
	\justifying
	%agregar referencias
	\bibliographystyle{apalike}
	%\nocite{*} % mostrar todas las referencias aunque no esten citadas
	\bibliography{bibliografia.bib}
\end{frame}



%%%%%%%%%%%%%%%%%%%%%%%%%%%%%%
%%%%%%%%%%%%%%%%%FIN DOCUMENTO
\end{document}
